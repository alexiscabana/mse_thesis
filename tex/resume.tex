\chapter*{Résumé}\label{cha:resume}
La simulation fonctionnelle forme une partie intégrante de la phase de test des projets de systèmes spatiaux. Pour le sous-système de communication du vaisseau spatial Dream Chaser, une fonction de type \textit{snapshot} a dû être développée pour son système de test logiciel-dans-la-boucle, similairement à des programmes de virtualisation. En utilisant des concepts existants, cette thèse démontre comment un instantané du logiciel de vol a pu être effectué, sans modifier son code, à travers l'injection de code visant à enregistrer l'état de ses fils d'exécution et en cataloguant ses variables entre deux éditions des liens. Après la production d'un artefact au format défini, l'environement de test a pu être entièrement restauré en manipulant les cadres de piles d'exécution pour reconstruire les fils de vol, parallèlement à la reconstruction des modules de simulation. Des résultats sont ensuite présentés pour démontrer la conformité de la fonctionalité aux requis de performance du client.