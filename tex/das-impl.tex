{
\setlength{\parindent}{2em}
\chapter{DAS Snapshotting - Implementation Analysis}\label{cha:das-impl}
Needs its own section, because big!
\section{FSW analysis}
- what needs to be saved?
\section{Black-box Symbol Access}
- Creation of the DAS state manager + symbol catalog
- how to access everything (talk about dynamic library loading, elf dumps, double linking etc.)
\section{Multithreaded Execution Backup and Restore}
tried several techniques to save/restore execution stacks:
- sufficient conditions to be able to restore a task (w or w/o touching exec stack)
- different strategies, ultimately bounded by different reasons 
- strat 1 : just replay the thread until the first waitUntilNextPeriod() (there's 4 possible states : not created yet, created but not once had the "go", waiting in a waitUntilNextPeriod(), created but deleted)
- strat 2 : restart and stub functions, can do it with ELF-HOOK https://www.codeproject.com/Articles/70302/Redirecting-functions-in-shared-ELF-libraries
	Redirection of the OSApiWrapper calls, Extremely useful, because our library is a dynamic object, which means its calls to library functions are always resolved at runtime by looking in the .rel.plt (relocation, Procedure Linkage Table)
- strat 3 complicated version of longjmp: how to save execution (asm, disassembly of code, save sp, etc.% https://cs.brown.edu/courses/cs033/docs/guides/x64_cheatsheet.pdf)
Depends on the ABI
\section{Design Constraints}
\subsection*{Shared Object Considerations}\label{sec:dynlib-considerations}
- Dynamic library loading  https://eli.thegreenplace.net/2011/08/25/load-time-relocation-of-shared-libraries
- address at which the library is loaded (why always 0x00007fffXXXXXXXX?) https://unix.stackexchange.com/questions/509607/how-a-64-bit-process-virtual-address-space-is-divided-in-linux
In this case, application continerization is not possible, because we are a library.

}