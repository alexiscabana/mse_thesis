{
\setlength{\parindent}{2em}
\chapter{State of the Art}\label{cha:state-of-the-art}
Saving the state of a running application to a file has been a very well-known challenge in the world of computer science, one that has its roots back to when the computers became powerful enough to run complicated programs. With the rise of local networks and the Ethernet protocol in the 1970s, an increasing amount of processing units could be linked together through a local network in order to improve the execution speed of difficult computation \cite{book:andrews}. This gave rise to the field of distributed computing, where scalability, parallelization of algorithms and efficient inter-node communication are among the very active topics of research.

Of course, the more complicated a system is, the more failure-prone it becomes. This is why it's imperative for designers of parallel algorithms to be careful when programming their application. It has to be made in such a way that it stays tolerant to eventual faults in the computing nodes of the network. For instance, one can think of the algorithms involved in numerical weather prediction software, that theoretically never finish as long as updated data is fed into the system. It is important for the forecasting industry not to loose any of the results that were previously solved if an unfortunate event occurs. \textit{Fault-tolerant computing} is the research field that finds solutions to this problem in multiple ways. In the context of distributed computing, one of the ways to mitigate the problem is to use a checkpoint and restart mechanism.

This technique allows the program to save itself while its running, and to restart at a previous checkpoint if processing stops. A "save" can take many forms, and its content is ultimately decided by the makers of the program. Before a final design is produced, some questions need to be answered:
\begin{enumerate}
	\item \textbf{What needs to be saved?} There needs to be a clear understanding of the program and how it works. Is saving only the intermediate result of a computation considered a sufficient condition to be able to restore the program back to where it was? Do we have to save the entire state of the operating system? Of the entire computer? 
	\item \textbf{What are we saving?} This can be binary data, numerical data, text, etc. This is again highly dependent on the application. A suitable file format has to be used depending on the data to store.
	\item \textbf{How are we saving it?} How can a checkpoint take form? This depends on the content. Most of the time, this will be a file written to non-volatile memory. Again, the file format has a role to play.
	\item \textbf{How often do we need to save?} A checkpoint can take a lot of space, and that amount usually grows linearly with the number of execution threads. In huge systems, this is not a trivial question. In addition, not only does a checkpoint take up hard disk space, it also induces an overhead in the execution of the program. Depending on the desired granularity, saving the relevant data can take a significant amount of time. This is represented by $O_F$ in \autoref{fig:chkpt-scheme}. This factor is important, especially in big distributed systems where computing time is expensive. As an example, the Titan supercomputer in the United States racks up \$9 million USD in electricity bills yearly \cite{online:henn}.
	\item \textbf{How long does it take to restart?} Saving at checkpoints takes time, but so does recovery. \autoref{fig:chkpt-scheme} shows this with $R_F$. Another point to consider is how often we need to restart back. In the end, restarting to a past state must be as straightforward as possible.
\end{enumerate}
\begin{figure}[H]
	\centering
	\includesvg[width=0.9\linewidth]{svg/chkpt-copy}
	\caption{Checkpoint/restart as a stochastic renewal reward process.}
	\label{fig:chkpt-scheme}
	\source{\textit{High Performance Computing Systems with Various Checkpointing Schemes} \cite{misc:chkpt-scheme}}
\end{figure}

\subsection*{Checkpointing Schemes}
There are different checkpointing schemes that are adapted to different needs. On one hand, it is possible to checkpoint the application at predetermined intervals $\Delta t$ (i.e every minute). This is useful when applicable, because it puts an upper bound on the amount of data/time loss in a worst case scenario. However, this mitigation method is not always possible for every type of computation. 

The second approach is to do it sporadically. This can be used when it's impossible to predict the amount of time required for a given computation. Unfortunately, it also means that the user doesn't know exactly when checkpoints will occur nor can he upper-bound the maximum amount of data/time loss.

\subsection*{Applicability to BBPSim}
Why exactly can these concepts be useful in the case of a simulator like BBPSim? The checkpoint and restart technique is not only applicable to the distributed computing, it can be adapted to fit the needs of multiple kinds programs. At the very least, some of the concepts can be used to design a homemade fault-tolerant feature. In the following sections, some existing \textit{snapshotting} solutions in released software will be investigated. Using available source code, we will see that a checkpoint/restart feature can be implemented at different levels within the execution system. Subsequently, the potential applicability of each implementation will be evaluated. In the end, the analysis will extract a set of working ideas to gather inspiration for the save \& restore in BBPSim.


\section{VirtualBox}\label{sec:virtualbox}
This open-source software project backed by Oracle is well-known in the virtualization industry. It is a hypervisor, a type of program defined by Red Hat as a
\begin{shadedquotation}
	[...] software that creates and runs [one or more] \gls{VM}. A hypervisor, sometimes called a virtual machine monitor (VMM), isolates the hypervisor operating system and resources from the virtual machines and enables the creation and management of those VMs.\cite{online:redhat}
\end{shadedquotation}
Indeed, VirtualBox acts as a mediator between a guest OS (the \textit{virtualized} OS) and a host OS. It is labeled as a Type-2 hypervisor, meaning that it is actually a software layer that separates both operating system, as \autoref{fig:layerhyper} shows. VirtualBox is in charge of exposing computer utilities to the guest operating system, like CPU time, RAM allocation, driver and graphics card access, etc. Like most of its counterparts, the hypervisor offers on top of that the possibility of \textit{snapshotting} a virtual machine's current state in order to allow a future restore to exactly this state : the state of the drivers, running process scheduling information, even the screen graphical interface. The feature outputs a sizable file as a result.

\begin{wrapfigure}{r}{0.35\textwidth}
	\centering \scriptsize
	\vspace{-12pt}
	\includesvg[width=0.34\textwidth]{svg/hypervisor}
	\caption{Abstraction layers for a type-2 hypervisor.}
	\label{fig:layerhyper}
\end{wrapfigure}
This application being open-source, it's possible to dive deeper and investigate the code itself. We are mostly interested here at how VirtualBox handles the saving of guest OS processes and memory, since BBPSim doesn't have a graphical interface nor uses the typical PC peripherals like USB or the audio output.

In the \texttt{src/VBox/}

\section{Checkpoint and Restore in User Space}\label{sec:criu}
https://github.com/checkpoint-restore/criu
\section{Berkeley Lab Checkpoint/Restart for Linux}\label{sec:blcr}
https://crd.lbl.gov/departments/computer-science/class/research/past-projects/BLCR/
- see also virtualization technology : https://github.com/dmtcp/dmtcp (process-based)

%https://github.com/dmtcp/dmtcp


%super interesting : http://citeseerx.ist.psu.edu/viewdoc/download?doi=10.1.1.126.8121&rep=rep1&type=pdf
%https://www.usenix.org/legacy/publications/library/proceedings/usenix01/freenix01/full_papers/dieter/dieter_html/paper.html
}