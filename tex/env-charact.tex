{
\setlength{\parindent}{2em}
\chapter{Software Environment}\label{cha:sim-env}

\section{Requirements}
\subsection{Customer Requirements}
\subsection{User Requirements}
- user requirements by me (FSW as untouched as possible, etc.)

\section{BBPSim Characteristics}
\subsection{Execution Environment}
\subsection{Commands}
\subsection{Interaction Layers}
- Overview of software architecture of simulator (show picture of layers)
- explain pthread, posix, they're used as wrapper for mutex in deos.
\subsection{Shared Object Considerations}
- Dynamic library loading  https://eli.thegreenplace.net/2011/08/25/load-time-relocation-of-shared-libraries
- address at which the library is loaded (why always 0x00007fffXXXXXXXX?) https://unix.stackexchange.com/questions/509607/how-a-64-bit-process-virtual-address-space-is-divided-in-linux
In this case, application continerization is not possible, because we are a library.

\section{FSW Outline}
\subsection{Modules}
- anatomy of the typical C module( static global, functions)
- anatomy of typical task loop in embedded systems 
\subsection{DEOS Overview}
deos particularities with preemption
}