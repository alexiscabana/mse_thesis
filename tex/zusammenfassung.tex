\chapter*{Zusammenfassung}\label{cha:zusammenfassung}
Funktionale Simulationen bilden einen integralen Teil der Testphase in Raumfahrtprojekten. Für das Kommunikations-Subsystem des Raumfahrzeugs Dream Chaser musste eine Snapshotfunktion für den Software-in-the-Loop-Testrahmen entwickelt werden, ähnlich wie bei Programmen für virtuelle Maschinen. Unter Verwendung von Konzepten in vorhandener Software zeigt diese Arbeit, wie die Flugsoftware ohne Änderung ihres Codes, durch das Einfügen von Thread-Checkpointing-Code zur Kompilierungszeit und Katalogisierung ihrer Variablen zur Erstellungszeit überprüft werden konnte. Nachdem eine Artefaktdatei mit einem geeigneten Format erzeugt wurde, war es möglich, die Simulationumgebung vollständig wiederherzustellen, indem Execution-Stack-Frames manipuliert wurden, um die Flug-Threads zu rekonstruieren, während die Simulationsmodule neu aufgebaut wurden. Es zeigt sich dann, dass die Ergebnisse sowohl die Leistungsanforderungen des Kunden als auch die des Benutzer erfüllt wurden.